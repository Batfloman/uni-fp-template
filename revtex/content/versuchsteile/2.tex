\section{Bestimmung von Y}

In diesem Versuchsteil wird die Größe $Y$ durch ... ermittelt.
Dafür werden dies und das gemacht.

\subsection{Druchführung / Messung}

Wir machen natürlich alles so, wie es die Versuchsanleitung möchte.

\subsection{Auswertung / Analyse}

Wir haben die Daten aus Tabelle~\ref{tab:linear-fit-datenpunkte} und fitten eine einfache Gerade:

\begin{figure}[H]
	\includegraphics[width=0.9\linewidth]{../output/linear_fit/plot_example.png}
	\caption{Example linear Fit: wir fitten die Daten aus Tab.~\ref{tab:linear-fit-datenpunkte} mit einer Funktion $y = mx + b$ an, wobei in userem Fall $x \hat{~=~} \nu$ (Frequenz) und $y \hat{~=~} G$ (Gain) gilt. Wir vergleichen einen \texttt{orthogonal-distance-regression} Fit mit einem einfachen \texttt{least-squares} Fit (beides mithilfe von scipy).}
\end{figure}

Die Parameter des \texttt{least-squares} Fits ergeben:
$$
	m = \ensuremath{2.00(10)}\,,\qquad n = \ensuremath{-0.07(26)}\,
$$

Die Parameter des \texttt{odr} Fits ergeben:
$$
	m^\texttt{ord} = \ensuremath{2.01(11)}\,,\qquad n^\texttt{odr} = \ensuremath{-0.2(4)}\,
$$

Wir sehen, dass die Fehler auf die Parameter des \texttt{odr}-Fits größer sind, dies macht Sinn, da dieser auch die Unsicherheiten in $x$ berücksichtigt.
